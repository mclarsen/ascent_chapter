Ascent's control interface consists of two main components:
the API and actions.
%
The API is primarily for getting data in and out of Ascent,
and the actions describe what to do with the simulation data
and what the simulation expects in return.

\subsubsection{Ascent API}
Ascent's front facing API consistst of five calls:
open, pushlish, execute, info, and close.
%
Ascent supports multiple language binding including C, C++,
Fortran, and Python.

%\begin{itemize}
%  \item open: initializes Ascent
%  \item publish: set the simulaiton data set
%  \item execute: performs a set of actions on the data set
%  \item info: returns information back to the simulation
%  \item close: cleans up Ascent
%\end{itemize}

\begin{lstlisting}[language=C++,caption={This cool caption}]
conduit::Node options, actions, data_set, info;
// fill options, actions, and data_set
ascent::Ascent ascent;
ascent.open(options);
ascent.publish(data_set);
ascent.execute(actions);
ascent.info(info);
ascent.close();
\end{lstlisting}

``open'' initializes ascent with a number of options including the
MPI communicator, exception handling, and the actions file name.
%
``publish'' takes in a Conduit Node containing the computational
mesh described using Blueprint.
%
``execute'' also takes in a conduit node containing the actions to
perform.
%
Typically, the parameters to ``exectute'' are overrided with a user provided
file, which allows actions to be changed without re-compiling the simulation.
%
``info'' is the mechanism for getting data out from Ascent into the simulation.
%
The Node passed to ``info'' is populated with data that includes the results
of queries, allowing the the simulation to take actions based on the results.
%
``close'' directs Ascent to finalize execution.


\subsubsection{Ascent Actions API}
In Ascent, actions are a declarative interface that enables users to
control Ascent's exectution using five high-level concepts:
Pipelines, Scenes, Extracts, Queries, and Triggers.
%
Pipelines describe a series of data tranfomations, e.g.,
clipping and contouring.
%
Scenes describe the images to be rendered, consisting of one or more
plots (e.g., pseudocolor or volume rendering).
%
Extracts are a way to to get data out of Ascent and into other softeware
ecosystems, such as python.
%
Queries are a way to ask questions about the simulation data over time, and
the results are available to the simulation.
%
Triggers describe conditional actions based on simulation state and are a way
to create adaptive workflows inside of Ascent.

Under the hood, tools like ParaView and VisIt construct data-flow networks
to execute commands issued by the user through the GUI.
%
Abstractly, Ascents actions are a level below a GUI but at a higher level
than assembling a data flow network from scratch.
%
Constructing actions provides the user greater control and understanding
of exactly what is executing in situ, while still affording a higher level of
useability than programming directly in somnething like VTK.
%
Additionally, many filters in Ascent provide \textit{relative}
parameters so the user doesn't need to explicitly know filter parameter values
before the simulation executes.
%
For example, the slice filter accepts \textit{relative} offsets from the
center of the data set, i.e., a value of $(0,0,0)$ would place the origin
of the slice plane a the center of the data set bounding box.

%


