%Ascent's control interface consists of two main components:
%the API and actions.
%
%The API is primarily for getting data in and out of Ascent,
%and the actions describe what to do with the simulation data
%and what the simulation expects in return.

%\subsubsection{Ascent API}
Ascent's API consists of five calls:
%open, pushlish, execute, info, and close.
%
%Ascent supports multiple language binding including C, C++,
%Fortran, and Python.
\begin{itemize}
  \item \textbf{open} initializes Ascent with options including the
MPI communicator, exception handling, and the actions file name.
%
  \item \textbf{publish} is the method that allows a simulation code to publish its data
to Ascent.  
%
   \item \textbf{execute} specifics the Ascent actions to perform (see \S\ref{sec:capabilities}).  
%
%Typically, the parameters to ``exectute'' are overrided with a user provided
%file, which allows actions to be changed without re-compiling the simulation.
%
  \item \textbf{info} is the mechanism for getting data out from Ascent into the simulation.
%
The Conduit Node passed to ``info'' is populated with data that includes the results
of queries, allowing the the simulation to take actions based on the results.
%
  \item \textbf{close} directs Ascent to finalize execution.
\end{itemize}

Ascent usage typically consists of only four calls: open, publish, execute, and close.
%
These calls are available with multiple language bindings: C, C++, Fortran, and Python.
%
%
\begin{lstlisting}[language=C++,caption={Typical Ascent usage in C++}]
conduit::Node data_set, actions;
// fill data_set 
// fill actions 
ascent::Ascent ascent;
ascent.open();
ascent.publish(data_set);
ascent.execute(actions);
ascent.info(info);
ascent.close();
\end{lstlisting}

The major work in an Ascent integration is setting up the data passed to ``publish'' and
``execute.''
%
The data passed to ``publish'' (``data\_set'' in the example) is in the Conduit-Blueprint form, which was discussed in \S\ref{Blueprint}.
%
The format for the data passed to ``execute'' (``actions'' in the example) 
is the subject of the remainder of this section.
%


