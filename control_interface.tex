%Ascent's control interface consists of two main components:
%the API and actions.
%
%The API is primarily for getting data in and out of Ascent,
%and the actions describe what to do with the simulation data
%and what the simulation expects in return.

%\subsubsection{Ascent API}
Ascent's API consists of five calls:
%open, pushlish, execute, info, and close.
%
%Ascent supports multiple language binding including C, C++,
%Fortran, and Python.
\begin{itemize}
  \item \textbf{open} initializes Ascent.  It can optionally take arguments, for passing along
information such as the MPI communicator.
%
  \item \textbf{publish} is the method that enables a simulation code to pass (``publish'') its data
to Ascent.  
%
   \item \textbf{execute} specifies which Ascent actions (see \S\ref{sec:capabilities}) to perform.  
%
%Typically, the parameters to ``execute'' are overriden with a user provided
%file, which allows actions to be changed without re-compiling the simulation.
%
  \item \textbf{info} is the mechanism for getting data out from Ascent into the simulation.
%
%The Conduit Node passed to ``info'' is populated with data that includes the results
%of queries, allowing the the simulation to take actions based on the results.
%
  \item \textbf{close} directs Ascent to finalize execution.
\end{itemize}

Ascent usage typically consists of only four calls: open, publish, execute, and close.
%
These calls are available with multiple language bindings: C, C++, Fortran, and Python.
%
%
\begin{lstlisting}[language=C++,caption={\label{listing:ascent_api}Typical Ascent usage in C++.
The code for ``fill data\_set'' can be found in Listing~\ref{ex:blueprint}, while the code
for ``fill actions'' can be found in Listing~\ref{listing:actions}.}]
conduit::Node data_set, actions;
// fill data_set 
// fill actions 
ascent::Ascent ascent;
ascent.open();
ascent.publish(data_set);
ascent.execute(actions);
ascent.close();
\end{lstlisting}

The major work in an Ascent integration is setting up the data passed to ``publish'' and
``execute.''
%
The data passed to ``publish'' (``data\_set'' in the example) is in the Conduit-Blueprint form, which was discussed in \S\ref{Blueprint}.
%
The format for the data passed to ``execute'' (``actions'' in the example) 
is the subject of the remainder of this section.
%


There are two forms for specifying actions to Ascent.
%
One form is to write code that specifies actions.
%
This code would set up Conduit Nodes, and be passed in as the ``actions'' in the
code listing above.
%
The code would be part of the simulation code, and compiled into the simulation.
%
The other form is to encode the information in a YAML file.
%
In this case, the code for the simulation code would simply direct Ascent to read
the YAML file and get the directions for the actions to take from that file.
%
The advantage of this latter form is that is that it decouples the visualization
operations from the simulation code.
%
Specifically, the simulation can be compiled into binary form, and yet the visualization
operations to be performed can still be changed.
%
It also allows one person to do code integration (setting up publishing of data)
and other people to control the visualization without having to understand
the code integration.

Despite the advantages of the YAML approach, we demonstrate setting up actions with
the code interface where the visualization operations are compiled into the code.
%
The result of this code listing is to create the ``actions'' from Listing~\ref{listing:ascent_api}.

\begin{lstlisting}[language=C++,caption={\label{listing:actions}Setting up Ascent actions using the C++ bindings.
%
This listing is broken into three parts.  
%
The first part makes a pipeline named ``pl1'' with one filter named ``f1.''
%
The second part makes a scene named ``s1'' which is connected to pipeline ``pl1.''
%
The third part tells Ascent to add the pipeline, add the scene, and then execute both.
}]
// Part 1: make a pipeline
Node pipelines;
pipelines["pl1/f1/type"] = "contour";
Node contour_params;
contour_params["field"] = "pressure";
double iso_vals[2] = {0.2, 0.4};
contour_params["iso_values"].set_external(iso_vals,2);
pipelines["pl1/f1/params"] = contour_params;

// Part 2: make a scene to render the dataset
Node scenes;
scenes["s1/plots/p1/type"] = "pseudocolor";
scenes["s1/plots/p1/pipeline"] = "pl1";
scenes["s1/plots/p1/field"] = "pressure";
scenes["s1/image_prefix"] = "pressure_iso";

// Part 3: prepare Conduit node for Ascent's execute() method
Node actions;
Node &add_act = actions.append();
add_act["action"] = "add_pipelines";
add_act["pipelines"] = pipelines;
Node &add_act2 = actions.append();
add_act2["action"] = "add_scenes";
add_act2["scenes"] = scenes;
actions.append()["action"] = "execute";
\end{lstlisting}

