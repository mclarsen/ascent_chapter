Ascent's control interface consists of two main components:
the API and actions.
%
The API is primaraly for getting data in and out of Ascent,
and the actions describe what to do with the simulation data
and what the simulation expects in return.

\subsection{Ascent API}
Ascent's front facing API consistst of five calls:
open, pushlish, execute, info, and close.
%
Ascent supports multiple language binding including C, C++,
Fortran, and Python.

%\begin{itemize}
%  \item open: initializes Ascent
%  \item publish: set the simulaiton data set
%  \item execute: performs a set of actions on the data set
%  \item info: returns information back to the simulation
%  \item close: cleans up Ascent
%\end{itemize}

\begin{lstlisting}[language=C++,caption={This cool caption}]
conduit::Node options, actions, data_set, info;
// fill options, actions, and data_set
ascent::Ascent ascent;
ascent.open(options);
ascent.pubish(data_set);
ascent.exectute(actions);
ascent.info(info);
ascent.close();
\end{lstlisting}

``open'' initializes ascent with a number of options including the
MPI communicator, exception handling, and the actions file name.
%
``publish'' takes in a Conduit Node containing the computational
mesh described using Blueprint.
%
``execute'' also takes in a conduit node containing the actions to
perform.
%
Typically, the parameters to ``exectute'' are overrided with a user provided
file, which allows actions to be changed without re-compiling the simulation.
%
``info'' is the mechanism for getting data out from Ascent into the simulation.
%
The Node passed to ``info'' is populated with data that includes the results
of queries, allowing the the simulation to take actions based on the results.
%
``close'' directs Ascent to finalize execution.


\subsection{Ascent Actions}
In Ascent, actions are a declarative interface that enables users to
control Ascent's exectution using five high-level concepts:
Pipelines, Scenes, Extracts, Queries, and Triggers.
%
Pipelines describe a series of data tranfomations, e.g.,
clipping and contouring.
%
Scenes describe the images to be rendered, consisting of one or more
plots (e.g., pseudocolor or volume rendering).
%
Extracts are a way to to get data out of Ascent and into other softeware
ecosystems, such as python.
%
Queries are a way to ask questions about the simulation data over time, and
the results are available to the simulation.
%
Triggers describe conditional actions based on simulation state and are a way
to create adaptive workflows inside of Ascent.

Under the hood, tools like ParaView and VisIt construct data-flow networks
to execute commands issued by the user through the GUI.
%
Abstractly, Ascents actions are a level below a GUI but at a higher level
than assembling a data flow network from scratch.
%
Constructing actions provides the user greater control and understanding
of exactly what is executing in situ, while still affording a higher level of
useability than programming directly in somnething like VTK.
%
Additionally, many filters in Ascent provide \textit{relative}
parameters so the user doesn't need to explicitly know filter parameter values
before the simulation executes.
%
For example, the slice filter accepts \textit{relative} offsets from the
center of the data set, i.e., a value of $(0,0,0)$ would place the origin
of the slice plane a the center of the data set bounding box.

\subsubsection{Pipelines}
Pipelines allows users to describe a series of data transformations, also known as
filters, to execute on simulation data.
%
An example pipeline could be to create an isovolume of a field
and clip the results in half.
%
In Ascent, users can define as many pipelines as needed.
%
By default, all actions consume the default pipeline, which is the data
published by the simulation, but pipelines and all other actions can consume
the results of pipelines.

\subsubsection{Scenes}
Scenes allow users to specify how images are rendered.
%
Each scene consists of one or more plots, and Ascent supports volume,
pseudocolor, and mesh plots.
%
Plot types contain parameters such scalar fields, color tables, and scalar ranges,
in addition to the name of the pipeline to consume.
%
Scenes also can contain zero or more \textit{renders}, which contain
camera specifications, image dimensions, background and
forground colors, annotation controls.
%
Ascent creates a default camera based on bounding box of the data set and
is always facing the data set.
%
Renders inherit the default camera parameters, which simplifies creating
cameras that actually looks at the data set.
%
Ascent also provides simple controls to rotate the camera on the
sphere circumscribing the data set, although the user is free to set all
camera parameters explicitly.
%
Additionally, scenes can create Cinema~\cite{AhrensCinema} databases that
create a large set of images that can be explored after the simulation.

\subsubsection{Extracts}
Extracts are an escape hatch in Ascent that enable data to be sent
outside of Ascent.
%
Extracts can be a simple a saving data to HDF5 files or can be a gateway
to a larger workflow(e.g., ADIOS).
%
Ascent supports connections to the Python ecosystem through the Python and
Jupyter extracts~\cite{CyrusISAV}.
%
Python extracts execute custom analysis code provided by the user, and the
Jupyter extracts allow for incoming connections of Jupyter notebooks from a web
browser.
%

Jupyter notebooks is a promising direction for in situ.
%
One of in situ's greatest weaknesses is the it's reliance on a priori
knowledge, and one strategy to mitagate this weakness is interactivity.
%
Through the Jupyter notebook interface, uses can pause a running simulation
and interact with the data.
%
Additionally, Jupyter widgets enable fast prototyping of domain specific
GUIs.

\fix{Cyrus add more wisdom and python stuff.}

\subsubsection{Queries}
Queries in Ascent allow users to ask questions about simulation data,
or data from pipelines, through a Python like language.
%
The expression language that backs queries can perform math operations,
call functions, conditionals, and store the results into named variables.
%
Listing~\ref{simple_query} shows the declaration of a query that returns
the current simulation cycle and stores it into a variable name \textit{cycle},
and Listing~\ref{complex_query} shows a more complex example of a query that
calculates the entropy of a simulation field.
%
Subsequent queries can refence the previous results, and using this,
more complex queries can be created.
%
Queries are executed each time Ascent is called, and the resulting time
history can be saved or a accessed by the simulation.

\begin{lstlisting}[language=Python,caption={Examples of querying the simulation cycle}, label={simple_query}]
# add a simple query expression (q1)
queries["q1/params/expression"] = "cycle()"
queries["q1/params/name"] = "cycle"
\end{lstlisting}

\begin{lstlisting}[language=Python,caption={A more complex example of queries in Ascent}, label={complex_query}]
# add a more complex query expression (q2)
queries["q2/params/expression"] = "entropy(histogram(field('gyre'), num_bins=128))"
queries["q2/params/name"] = "entropy_of_gyre"
\end{lstlisting}


\subsubsection{Triggers}
