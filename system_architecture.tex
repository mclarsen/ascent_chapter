\subsection{Flow: A data-type agonistic data-flow based architecture}
\fix{
This text is essentially from the ISAV paper.
%
At Ascent's core is a simple data-flow library, named Flow, to efficiently compose
and execute filters.
%
Ascent’s Flow library is a C++ evolution of a Python data flow network
infrastructure~\cite{flow_reference}.
%
Flow supports declaration and execution of directed acyclic graphs
(DAGs) of filters.
%
Filters declare a minimal interface, which includes the number of
expected inputs and outputs, and a set of default parameters.
%
Flow uses a topological sort to ensure proper filter execution order,
tracks all intermediate results, and provides basic memory management capabilities.
}

There are three main components to Flow:
\begin{itemize}
  \item \textbf{Registry}: manages the lifetime of intermediate filter results
  \item \textbf{Graph}: contains the filter graph and manages the adding of filters
  \item \textbf{Workspace}: container for both the registry and filter graph
\end{itemize}

\paragraph{Registry}
The registry is a key-value data store used to manage the intermediate
results of filters inside the data-flow network.
%
Keys within the registry are reference counted, and data contained
inside the registry is deleted when the reference counts reach zero.
%
While the data associated with a key can be a pointer to any type,
the majority of the data stored in the registry are Conduit nodes
or VTK-m data sets.

\paragraph{Graph}
The graph.

\paragraph{Workspace}
The workspace.

\paragraph{Flow Filters}
There are fitlers.

\subsection{Node-level parallelism? Words to describe VTK-m and other components}

Component based architecture? Components include VTK-m, Devil Ray, perhaps things like ADIOS and other extracts.

\subsection{Distributed-Memory Execution}

\subsection{Extracts: optional python environment}
I feel like this subsection is about the ways to get to other things.
