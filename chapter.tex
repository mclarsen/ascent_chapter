%%%%%%%%%%%%%%%%%%%%% chapter.tex %%%%%%%%%%%%%%%%%%%%%%%%%%%%%%%%%
%
% sample chapter
%
% Use this file as a template for your own input.
%
%%%%%%%%%%%%%%%%%%%%%%%% Springer-Verlag %%%%%%%%%%%%%%%%%%%%%%%%%%
%\motto{Use the template \emph{chapter.tex} to style the various elements of your chapter content.}
\chapter{Ascent: A Flyweight In Situ Library for Exascale Simulations}
\label{ascent} % Always give a unique label
% use \chaptermark{}
% to alter or adjust the chapter heading in the running head

\abstract*{
In situ coupling poses a unique set of challenges to the design of analysis and
visualization infrastructures.
%
When running with a simulation, resources, such as time and memory are constrained.
%
As simulation applications port to accelerator-based HPC architectures, so
must our analysis tools.
%
Further, integrating into a simulation's build system adds additional complexity
becomes difficult with more dependencies.
%
Our current community tools were not designed with these set of issues in mind.
%
Ascent is an emerging is situ analysis and visualization infrastructure that
addresses these challenges.
%
From Ascent's inception, we have targeted in situ use cases and many-core
architectures.
%
In this paper, we will discuss in detail Ascent's design considerations, data and control interfaces, system architecture, and success stories.
}

This chapter describes the Ascent library for in situ visualization
and analysis.
%
Ascent was designed to deliver on two goals which are often in tension:
 (1) minimize encumbrance on simulation codes that incorporate Ascent,
i.e., a ``flyweight'' design, and 
(2) deliver maximum capability, especially for modern supercomputers,
%
In addition, Ascent built for production use --- it has documentation, 
examples, engages in modern software engineering practices, etc.

In terms of flyweight design, Ascent aims to minimize 
execution time, memory usage, binary size, and code integrations.
%
The first two, execution time and memory usage, benefit from directly
incorporating VTK-m~\cite{VTK-m} into Ascent.  
%
VTK-m's data model supports many array layouts (e.g., row major versus 
column major, array of structures versus structures of arrays) 
in a zero-copy manner.
%
Further, VTK-m provides native support for many-core architectures,
including Nvidia GPUs, various Intel architectures, 
and planned support for AMD GPUs.
%
As a result, Ascent is able to perform its algorithms very quickly,
since it can make full use of the underlying hardware architecture.
%
Statement about binary size.
%
Statement about code integrations that references API section.

Statement about capabilities.  Emphasis on being an integrator.
This is where Jupyter, triggers, etc. get mentioned.  Also emphasize
exascale.
Ascent has a rich feature set and integrates with other ecosystems.

\begin{itemize}
  \item Supports common visualization operations such as slicing and dicing your data.
  \item Supports ray traced surface and volume rendering as well as simulated radiography.
  \item Supports Queries for getting quantitative answers to questions
  \item Supports Triggers for adaptive visualization and analysis
  \item Outputs images sequences and Cinema image databases
  \item Outputs extracts for exporting data to HDF5, ADIOS, Python and Jupyter
  \item Support interactive visualization and analysis with Jupyter and Catalyst
  \item Supports C/C++, Python and Fortran language bindings
\end{itemize}

Statement about Devil Ray, but a forward reference a section in success stories --- success as an integrator.



\section{The Ascent In Situ Infrastructure}
\label{ascent_overview}
%Use the template \emph{chapter.tex} together with the document class SVMono (monograph-type books) or SVMult (edited books) to style the various elements of your chapter content conformable to the Springer Nature layout.

%List how ascent makes things easy.

\section{Design Considerations}
\label{ascent_design_considerations}

\subsection{Reduce required dependencies and adhere to a lightweight memory footprint}
Software complexity. Ascent is less complex.

\subsection{Target exascale architectures: maximize constrained resources}
Ascent's primary in situ use case is tightly-coupled, i.e.,
Ascent shares the same computational resources (proximity) and
memory space as the simulaiton (access).
%
All core functionality in Ascent, transforming data
and making pictures, executes on the same hardware as the
simulation through portably performance abstractions such as
VTK-m and RAJA.
%
This enables Ascent to minimize data movement by directly accessing
the simulation data memory without making copies of the data.
%

Memory and time are both constrained resources in situ.

%
Simulations often consume almost all availble system memory,
leaving only a small fraction to in situ infrastructures.
%
It is imperative that Ascent uses memory as efficiently as possible
as not to exceed system memory.
%
To that end, Ascent can directly access simulation memory, whether it be
on the CPU or GPU.
%
While important, direct simulation memory access is only half
the battle of memory efficiency.
%
Visualization pipelines transforms data(e.g., isocontours) from
one form to another, and efficiently managing the memory consumed
by intermediate results is equally important.
%
\fix{possible example: contour of turbulence simulation on
structured grid can create a dataset larger than the original.}
%
In Ascent, we free memory used by intermediate results as soon
as they are not needed by downstream transforms.
%


%
Supercomper architectures have significanlty changed since
%
Current community visualization tools, such as ParaView and VisIt, were originally
designed to execute using one MPI rank per CPU core.
%

All core functionality(i.e., transforming data and making pictures) in
Ascent targets exascale architectures.

VTK-m, devil ray

\subsection{Provide simple and flexible data and control interfaces}

\subsection{Create tools to address in situ resource constraints , e.g., a flexible trigger system that allows users to express what is important}

\subsection{Simplify connection to other ecosystems, e.g., python, custom analysis, and Juptyer Notebooks}



\section{Data Interface}
\label{ascent_data}
\subsection{Conduit: A foundation for in-memory data exchange}
\label{sec:conduit}
Conduit is a library that provides an intuitive model for describing
hierarchical scientific data in C++, C, Fortran, and Python.
%
The API is a JSON-inspired data model for describing hierarchical
in-core scientific data.
%
While primarily used for data coupling between packages in-core,
Conduit also provides easy access to serialization and I/O functions.
%
In this section, we will discuss some of the basics of Conduit
and describe the concepts needed to couple simulation mesh
data structures using Conduit's Mesh Blueprint.

\fix{At the end, we should mention how simulations codes are using conduit as
a central data store, and for checkpoints. It also might be nice to
illustrate some other interesting use cases like neurons, or diffing nodes
to update rendering states in wolf vision. The limits of using using Conduit
are only bounded by the imagination of the person who wields Conduit.}

\subsubsection{Conduit Basics}
The primary Conduit data structure is a Node.
%
Nodes stores data through a key-value interface.
%
Figure~\ref{ex:1} shows how to store a string ``value'' into a Node
with the key ``key''.

\begin{figure}
\begin{tabular}{cc}
  \begin{minipage}{.5\textwidth}
  \centering
    \begin{lstlisting}[language=C++]
Node n;
n["key"] = "value";
n.print();
    \end{lstlisting}
  \end{minipage}
  &
  \begin{minipage}{.5\textwidth}
  \centering
  \begin{lstlisting}[language=C++]
{
  "key": "value"
}
  \end{lstlisting}
  \end{minipage}
\end{tabular}
\caption{\label{ex:1}On the left, an example of storing a string inside a Conduit Node. On the right, the JSON equivalent.}
\end{figure}

Data in Nodes can be created and accessed through hierarchical keys,
and the key string looks much like a UNIX directory structure.
%
Key paths are completely up to the user.
%
Figure~\ref{ex:2} illustrates creating a Node hierarchy
and storing a floating point value in a leaf Node.
%
In this example, several Nodes are actually created with parent child
relationships defined by the key.
%
Node $n$ has a child Node $dir1$, which in turn has a child $dir2$,
and the tree ends at the leaf Node $val1$, which stores the data.
%
Nodes can contain many basic data types such as strings,
floating point values, and integers.

\begin{figure}
\begin{tabular}{cc}
  \begin{minipage}{.5\textwidth}
  \centering
    \begin{lstlisting}[language=C++]
Node n;
n["dir1/dir2/val1"] = 100.5;
n.print();
    \end{lstlisting}
  \end{minipage}
  &
  \begin{minipage}{.5\textwidth}
  \centering
  \begin{lstlisting}[language=C++]
{
  "dir1" :
  {
    "dir2" :
    {
      "val1": 100.5
    }
  }
}
  \end{lstlisting}
  \end{minipage}
\end{tabular}
\caption{\label{ex:2}On the left, an example of using a hierarchical key to store a number. On the right, the JSON equivalent.}
\end{figure}

Nodes can also contain arrays, and the $set$ method
copies the values from an array into a Node.
%
Figure~\ref{ex:3} shows how to set a Node to the contents of an array.
%
Alternatively, the $set_external$ copies only the pointer(i.e., a shallow copy),
and any change to the underlying array would be reflected in both the original
array and the data contained inside the Node.
%
Using $set_external$ is desirable for large data, such as simulation mesh data,
and in environments where memory is constrained.

\begin{figure}
\begin{tabular}{cc}
  \begin{minipage}{.5\textwidth}
  \centering
    \begin{lstlisting}[language=C++]
const int size = 4;
int A[size] = {0, 1, 2, 3};
Node n;
n["my_array"].set(A, size);
n.print();
    \end{lstlisting}
  \end{minipage}
  &
  \begin{minipage}{.5\textwidth}
  \centering
  \begin{lstlisting}[language=C++]
{
  "my_array" : [0, 1, 2, 3]
}
  \end{lstlisting}
  \end{minipage}
\end{tabular}

\caption{\label{ex:3}On the left, an example of setting the value of a Node to an Array. On the right, the JSON equivalent.}
\end{figure}

\subsection{Mesh Blueprint: An in-memory mesh description interface co-designed with simulation applications
System Architecture}

The flexibility of the Conduit Node allows it to be used to represent a
wide range of scientific data.
%
Unconstrained, this flexibly can lead to
many application specific choices for common types of data that could
potentially be shared between applications.

The goal of Blueprint is to help facilitate a set of shared higher-level
conventions for using Conduit Nodes to hold common simulation data structures.
%
The Blueprint library in Conduit provides methods to verify if a Conduit
Node instance conforms to known conventions, which we call protocols.
%
It also provides property and transform methods that can be used on conforming Nodes.

Many taxonomies and concrete mesh data models have been developed to allow
computational meshes to be used in software.
%
Blueprint’s conventions for representing mesh data were formed by negotiating
with simulation application teams at LLNL and from a survey of existing
projects that provide scientific mesh-related APIs including: ADIOS, Damaris,
EAVL, MFEM, Silo, VTK, VTKm, and Xdmf.
%
Blueprint’s mesh conventions are not a replacement for existing mesh data
models or APIs.
%
Our explicit goal is to outline a comprehensive, but small set of options
for describing meshes in-core that simplifies the process of adapting data
to several existing mesh-aware APIs.

Blueprint covers a wide range of mesh descriptions, and
Blueprint uses four concepts to describe meshes:

\begin{itemize}
  \item Coordinate Sets
  \item Topologies
  \item Fields
  \item Domain Decomposition Information
\end{itemize}

Coordinate sets described coordinate systems in 1D, 2D, and 3D, and
can be specified in cartesian, spherical, or cylindrical frames of reference.
%
In addition to explicit coordinate sets, compact implicit representations,
such as uniform and rectilinear, are also supported.
%
Topologies describe the topological structure of the mesh elements.
%
As with coordinate sets, both implicit(e.g., uniform) and explicit
(e.g., completely unstructured)topologies are supported.
%
Fields describe the data associated with the mesh, and Blueprint supports
fields associated with vertices or elements.
%
Fields can be scalars or have multiple components.
%
Additionally, Blueprint supports the description of high-order
topologies and fields, which are becoming increasingly common.

A Blueprint data set minimally needs a topology and a coordinate system,
but Blueprint supports having any number of topologies and coordinate
systems.
%
Figure~\ref{ex:blueprint} shows a Blueprint description of a uniform mesh.

\begin{figure}
\begin{lstlisting}[language=C++]
Node mesh;
// 10x10x10 uniform cooridinate system
mesh["coordsets/my_coords/type"]="uniform";
mesh["coordsets/my_coords/dims/i"] = 10;
mesh["coordsets/my_coords/dims/j"] = 10;
mesh["coordsets/my_coords/dims/k"] = 10;

// optional origin
mesh["coordsets/my_coords/origin/x"]= -10;
mesh["coordsets/my_coords/origin/y"]= -10;
mesh["coordsets/my_coords/origin/z"]= -10;

mesh["coordsets/my_coords/spacing/dx"]= 1.0;
mesh["coordsets/my_coords/spacing/dy"]= 1.0;
mesh["coordsets/my_coords/spacing/dz"]= 1.0;

mesh["topologies/my_topo/type"] = "uniform";
mesh["topologies/my_topo/coordset"] = "my_coords";
\end{lstlisting}
\caption{\label{ex:blueprint}An example of a specifying a $10^3$ uniform grid in Blueprint.}
\end{figure}


\section{Control Interface}
\label{ascent_control}
%Ascent's control interface consists of two main components:
%the API and actions.
%
%The API is primarily for getting data in and out of Ascent,
%and the actions describe what to do with the simulation data
%and what the simulation expects in return.

%\subsubsection{Ascent API}
Ascent's API consists of five calls:
%open, pushlish, execute, info, and close.
%
%Ascent supports multiple language binding including C, C++,
%Fortran, and Python.
\begin{itemize}
  \item \textbf{open} initializes Ascent with options including the
MPI communicator, exception handling, and the actions file name.
%
  \item \textbf{publish} is the method that allows a simulation code to publish its data
to Ascent.  
%
   \item \textbf{execute} specifies the Ascent actions to perform (see \S\ref{sec:capabilities}).  
%
%Typically, the parameters to ``exectute'' are overrided with a user provided
%file, which allows actions to be changed without re-compiling the simulation.
%
  \item \textbf{info} is the mechanism for getting data out from Ascent into the simulation.
%
The Conduit Node passed to ``info'' is populated with data that includes the results
of queries, allowing the the simulation to take actions based on the results.
%
  \item \textbf{close} directs Ascent to finalize execution.
\end{itemize}

Ascent usage typically consists of only four calls: open, publish, execute, and close.
%
These calls are available with multiple language bindings: C, C++, Fortran, and Python.
%
%
\begin{lstlisting}[language=C++,caption={Typical Ascent usage in C++}]
conduit::Node data_set, actions;
// fill data_set 
// fill actions 
ascent::Ascent ascent;
ascent.open();
ascent.publish(data_set);
ascent.execute(actions);
ascent.info(info);
ascent.close();
\end{lstlisting}

The major work in an Ascent integration is setting up the data passed to ``publish'' and
``execute.''
%
The data passed to ``publish'' (``data\_set'' in the example) is in the Conduit-Blueprint form, which was discussed in \S\ref{Blueprint}.
%
The format for the data passed to ``execute'' (``actions'' in the example) 
is the subject of the remainder of this section.
%




\section{System Architecture}
\label{ascent_architecture}
\subsection{Flow: A data-type agonistic data-flow based architecture}
At Ascent's core is a simple data-flow library, called Flow, that
composes and executes filters, the basic unit of execution in Ascent.
%
Flow is an evolution of a Python data-flow network
~\cite{flow_reference}, but unlike its ancestor, Flow is a C++
library.
%
Flow supports composing and executing directed acyclic graphs
(DAGs) composed of filters~\cite{LarsenAscent}.

There are three components to Flow:
\begin{itemize}
  \item \textbf{Registry}: manages the lifetime of intermediate filter results
  \item \textbf{Graph}: contains the filter graph and manages the adding of filters
  \item \textbf{Workspace}: container for both the registry and filter graph
\end{itemize}

\paragraph{Registry}
The registry is a key-value data store used to manage the intermediate
results of filters inside the data-flow network.
%
Keys within the registry are reference counted, and data contained
inside the registry is deleted when the reference counts reach zero.
%
While the data associated with a key can be a pointer to any type,
the majority of the data stored in the registry are Conduit nodes
or VTK-m data sets.

\paragraph{Graph}
The graph interface supports main operations: adding filters
and connecting filters together.
%

\paragraph{Workspace}
The workspace is a container for both the graph and the registry,
and the workspace is responsible for creating an execution for plan
for the DAG.
%
\fix{Cyrus: make sure these words are true.}
Flow uses a topological sort to ensure proper filter execution order,
tracks all intermediate results, and provides basic memory management capabilities.
%
Multiple workspaces can co-exist, and in fact, Ascent uses a Flow workspace
to evaluate expressions within the Ascent runtime.

\paragraph{Flow Filters}
Flow filters are the basic unit of execution inside of Ascent, and
almost all functionality inside of Ascent is implemented as a Flow filter.
%
Filters declare an interface, i.e., how many inputs a filter has and
if there is an output.
%
Inside Ascent, typical inputs and outputs are data sets.
%
Additionally, filters are passed a set of parameters inside of a Conduit
node.
%

\subsection{Node-level parallelism? Words to describe VTK-m and other components}

Component based architecture? Components include VTK-m, Devil Ray, perhaps things like ADIOS and other extracts.

\subsection{Distributed-Memory Execution}

\subsection{Extracts: optional python environment}
I feel like this subsection is about the ways to get to other things.


\section{Success Stories}
\label{ascent_success}
\subsection{In situ visualization of an Inertial Confinement Fusion (ICF) simulation}

Ascent was used to visualize the results of an unprecedented 3D simulation
of two-fluid mixing in a spherical geometry to better understand hydrodynamic
instability and the transition to turbulence process that is important to
the field of inertial confinement fusion and High Energy Density (HED)
Physics. High resolution simulations of instability growth are not practical
for routine use, so high resolution simulations like this help guide the
development of sub-grid models that capture instability effects with much
less computational cost, which are used for ICF calculations.

\begin{figure}
\centering
\includegraphics[trim={ 0 8cm 0 7cm},width=0.9\textwidth]{images/mixing_ball}
\caption{\label{img:icf}
This image is of an idealized Inertial Confinement
Fusion (ICF) simulation of a Rayleigh–Taylor instability
with two fluids mixing in a spherical geometry.
An isovolume filter was used to show only the mixing region of the heavy and
light fluids, and then a clip filter was added to show the internal of the sphere.
}
\end{figure}

The simulation was run on the Lawrence Livermore National Laboratory's (LLNL)
Sierra system, a 125 Petaflop peak system from IBM that has 4,320 nodes,
each with 2 IBM POWER9 processors, 4 NVIDIA Tesla V100 GPUs, 320 GiB of
fast memory (256 GiB DDR4 memory and 64 GiB HBM4), and 1.6TB of NVMe
memory. The specific simulation was a 97.8 billion element simulation
run across 16,384 GPUs on 4,096 compute nodes. The simulation application
used CUDA via RAJA to run on the GPUs. The time-varying evolution of the
mixing was visualized in situ with Ascent, also leveraging 16,384 GPUs.
The last time step was also exported by Ascent to the parallel file
system for detailed post-hoc visualization using VisIt\cite{VisIt}. The simulation
data was accessed by Ascent directly from the GPU memory, eliminating any
extra data copies.
Figure~\ref{img:icf} shows one of the many images generated in situ during
this run.

\subsection{MARBL Simulation Integration}
Ascent has been integrated and released with LLNL's MARBL~\cite{marbl} simulation code,
a new next-generation multi-physics code currently under development.
%
One of MARBL's components is a high-order finite element solver build on
MFEM~\cite{mfem}, and Ascent supports the MFEM data model.
%
In order to leverage Ascent's visualization capabilities, Ascent refines
the high-order elements to low-order(i.e., linear elements).
%
Ascent can be activated through the simulation's input deck, and in addition
to adding in situ visualization capabilities, Ascent can be used to save out
the mesh and only the fields that the user specifies.
%
Previously, MARBL only saved out full checkpoints, and only saving a subset of the
data allows users to save data for post-hoc analysis at much higher temporal resolution.

As a new code, simulation validation plays an important role, and one method
for simulation validation is comparing the results of experimental data to the
simulated experiments.
%
One such experiment is a radiation driven Kelvin-Helmholtz shear
layer experiment~\cite{hurricane2009high}.
%
The experiment captured radiographs of the instability as it was driven through
the material. Comparing experimental radiographs with radiographs created from
the simulation data is a useful simulation validation approach.
%
MARBL ran a simulation of this problem using 2304 MPI ranks for 120 hours.
%
Figure~\ref{img:radkh} shows a volume rendering and simulated radiograph
generated as a result of this simulation.
%
\begin{figure}
\centering
\includegraphics[width=0.4\textwidth]{images/radkh}
\includegraphics[width=0.4\textwidth]{images/radkh_xray}
  \caption{\label{img:radkh}A volume rendering(left) and simulated radiograph(right) created by
Ascent during a Kelvin-Helmholtz simulation.}
\end{figure}

%\subsection{Integrations}
%
%\begin{itemize}
%\item MARBL
%\item ARES
%\item AMRex: Warpx, Pele, NYX
%\item SW4
%\end{itemize}




