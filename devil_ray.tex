High-order finite element simulations, like MARBL, are becoming more
common as supercomputing architectures continue to become more
heterogenous.
%
One reason for this is that, by simply adjusting the polynomial order
of the mesh elements, high-order simulations
can tune the FLOPS/byte ratio to optimize performance on a specific architecture.
%
Additionally, high-order finite element methods can improve the overall solution accuracy.
%
Analysis and visualization plays a key role in understanding, debugging,
and communicating simulation results, but analysis and visualization frameworks
have traditionally only targeted low-order meshes.
%
The geometry of a high-order dataset is traditionally processed by converting
to a low-order approximation, and in order to maintain the accuracy of
the solution, low-order refinement can increase the size of data representation
by many orders of magnitude.
%
To make matters worse, the best refinement level is unknown,
and the error propagated by the refinement is not easily understood by users.
%
With the rise of in situ analysis, low-order refinement imposes additional memory
and time constraints on the simulation.
%

By default, Ascent performs low-order refines to visualize high-order meshes,
but Ascent has recently integrated Devil Ray, a library for natively ray tracing
high-order element meshes.
%



