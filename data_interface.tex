\subsection{Conduit: A foundation for in-memory data exchange}
Everything is a conduit node.

\subsubsection{Conduit Basics}
%

\subsection{Mesh Blueprint: An in-memory mesh description interface co-designed with simulation applications
System Architecture}

The flexibility of the Conduit Node allows it to be used to represent a
wide range of scientific data.
%
Unconstrained, this flexibly can lead to
many application specific choices for common types of data that could
potentially be shared between applications.

The goal of Blueprint is to help facilite a set of shared higher-level
conventions for using Conduit Nodes to hold common simulation data structures.
%
The Blueprint library in Conduit provides methods to verify if a Conduit
Node instance conforms to known conventions, which we call protocols.
%
It also provides property and transform methods that can be used on conforming Nodes.

Many taxonomies and concrete mesh data models have been developed to allow
computational meshes to be used in software.
%
Blueprint’s conventions for representing mesh data were formed by negotiating
with simulation application teams at LLNL and from a survey of existing
projects that provide scientific mesh-related APIs including: ADIOS, Damaris,
EAVL, MFEM, Silo, VTK, VTKm, and Xdmf.
%
Blueprint’s mesh conventions are not a replacement for existing mesh data
models or APIs.
%
Our explicit goal is to outline a comprehensive, but small set of options
for describing meshes in-core that simplifies the process of adapting data
to several existing mesh-aware APIs.

\subsection{DataObject}

I feel like we should list the different data models we currenlty support and how we
move from one to the other.
%
This might go into system architecture.

Ascent supports mutliple data representations, including Blueprint (both low
order and high order meshes), and VTK-m.
%
The Data Object is an abstraction that encapsulates data adapter code in order
to freely convert between compatible representations.
%
Filters in Ascent can ask for whatever data representation that they
support.


