\subsection{Conduit: A foundation for in-memory data exchange}
\label{sec:conduit}
Conduit is a library that provides an intuitive model for describing
hierarchical scientific data in C++, C, Fortran, and Python.
%
The API is a JSON-inspired data model for describing hierarchical
in-core scientific data.
%
While primarily used for data coupling between packages in-core,
Conduit also provides easy access to serialization and I/O functions.
%
In this section, we will discuss some of the basics of Conduit
and describe the concepts needed to couple simulation mesh
data structures using Conduit's Mesh Blueprint.

\fix{At the end, we should mention how simulations codes are using conduit as
a central data store, and for checkpoints. It also might be nice to
illustrate some other interesting use cases like neurons, or diffing nodes
to update rendering states in wolf vision. The limits of using using Conduit
are only bounded by the imagination of the person who wields Conduit.}

\subsubsection{Conduit Basics}
The primary Conduit data structure is a Node.
%
Nodes stores data through a key-value interface.
%
Figure~\ref{ex:1} shows how to store a string ``value'' into a Node
with the key ``key.''

\begin{figure}
\begin{tabular}{cc}
  \begin{minipage}{.5\textwidth}
  \centering
    \begin{lstlisting}[language=C++]
Node n;
n["key"] = "value";
n.print();
    \end{lstlisting}
  \end{minipage}
  &
  \begin{minipage}{.5\textwidth}
  \centering
  \begin{lstlisting}[language=C++]
{
  "key": "value"
}
  \end{lstlisting}
  \end{minipage}
\end{tabular}
\caption{\label{ex:1}On the left, an example of storing a string inside a Conduit Node. On the right, the JSON equivalent.}
\end{figure}

Data in Nodes can be created and accessed through hierarchical keys,
and the key string looks much like a UNIX directory structure.
%
Key paths are completely up to the user.
%
Figure~\ref{ex:2} illustrates creating a Node hierarchy
and storing a floating point value in a leaf Node.
%
In this example, several Nodes are actually created with parent child
relationships defined by the key.
%
Node $n$ has a child Node $dir1$, which in turn has a child $dir2$,
and the tree ends at the leaf Node $val1$, which stores the data.
%
Nodes can contain many basic data types such as strings,
floating point values, and integers.

\begin{figure}
\begin{tabular}{cc}
  \begin{minipage}{.5\textwidth}
  \centering
    \begin{lstlisting}[language=C++]
Node n;
n["dir1/dir2/val1"] = 100.5;
n.print();
    \end{lstlisting}
  \end{minipage}
  &
  \begin{minipage}{.5\textwidth}
  \centering
  \begin{lstlisting}[language=C++]
{
  "dir1" :
  {
    "dir2" :
    {
      "val1": 100.5
    }
  }
}
  \end{lstlisting}
  \end{minipage}
\end{tabular}
\caption{\label{ex:2}On the left, an example of using a hierarchical key to store a number. On the right, the JSON equivalent.}
\end{figure}

Nodes can also contain arrays, and the $set$ method
copies the values from an array into a Node.
%
Figure~\ref{ex:3} shows how to set a Node to the contents of an array.
%
Alternatively, the $set_external$ copies only the pointer(i.e., a shallow copy),
and any change to the underlying array would be reflected in both the original
array and the data contained inside the Node.
%
Using $set_external$ is desirable for large data, such as simulation mesh data,
and in environments where memory is constrained.

\begin{figure}
\begin{tabular}{cc}
  \begin{minipage}{.5\textwidth}
  \centering
    \begin{lstlisting}[language=C++]
const int size = 4;
int A[size] = {0, 1, 2, 3};
Node n;
n["my_array"].set(A, size);
n.print();
    \end{lstlisting}
  \end{minipage}
  &
  \begin{minipage}{.5\textwidth}
  \centering
  \begin{lstlisting}[language=C++]
{
  "my_array" : [0, 1, 2, 3]
}
  \end{lstlisting}
  \end{minipage}
\end{tabular}

\caption{\label{ex:3}On the left, an example of setting the value of a Node to an Array. On the right, the JSON equivalent.}
\end{figure}

\subsection{Mesh Blueprint: An in-memory mesh description interface co-designed with simulation applications
System Architecture}

The flexibility of the Conduit Node allows it to be used to represent a
wide range of scientific data.
%
Unconstrained, this flexibly can lead to
many application specific choices for common types of data that could
potentially be shared between applications.

The goal of Blueprint is to help facilitate a set of shared higher-level
conventions for using Conduit Nodes to hold common simulation data structures.
%
The Blueprint library in Conduit provides methods to verify if a Conduit
Node instance conforms to known conventions, which we call protocols.
%
It also provides property and transform methods that can be used on conforming Nodes.

Many taxonomies and concrete mesh data models have been developed to allow
computational meshes to be used in software.
%
Blueprint’s conventions for representing mesh data were formed by negotiating
with simulation application teams at LLNL and from a survey of existing
projects that provide scientific mesh-related APIs including: ADIOS, Damaris,
EAVL, MFEM, Silo, VTK, VTKm, and Xdmf.
%
Blueprint’s mesh conventions are not a replacement for existing mesh data
models or APIs.
%
Our explicit goal is to outline a comprehensive, but small set of options
for describing meshes in-core that simplifies the process of adapting data
to several existing mesh-aware APIs.

Blueprint covers a wide range of mesh descriptions, and
Blueprint uses four concepts to describe meshes:

\begin{itemize}
  \item Coordinate Sets
  \item Topologies
  \item Fields
  \item Domain Decomposition Information
\end{itemize}

Coordinate sets described coordinate systems in 1D, 2D, and 3D, and
can be specified in cartesian, spherical, or cylindrical frames of reference.
%
In addition to explicit coordinate sets, compact implicit representations,
such as uniform and rectilinear, are also supported.
%
Topologies describe the topological structure of the mesh elements.
%
As with coordinate sets, both implicit(e.g., uniform) and explicit
(e.g., completely unstructured)topologies are supported.
%
Fields describe the data associated with the mesh, and Blueprint supports
fields associated with vertices or elements.
%
Fields can be scalars or have multiple components.
%
Additionally, Blueprint supports the description of high-order
topologies and fields, which are becoming increasingly common.

A Blueprint data set minimally needs a topology and a coordinate system,
but Blueprint supports having any number of topologies and coordinate
systems.
%
Figure~\ref{ex:blueprint} shows a Blueprint description of a uniform mesh.

\begin{figure}
\begin{lstlisting}[language=C++]
Node mesh;
// 10x10x10 uniform cooridinate system
mesh["coordsets/my_coords/type"]="uniform";
mesh["coordsets/my_coords/dims/i"] = 10;
mesh["coordsets/my_coords/dims/j"] = 10;
mesh["coordsets/my_coords/dims/k"] = 10;

// optional origin
mesh["coordsets/my_coords/origin/x"]= -10;
mesh["coordsets/my_coords/origin/y"]= -10;
mesh["coordsets/my_coords/origin/z"]= -10;

mesh["coordsets/my_coords/spacing/dx"]= 1.0;
mesh["coordsets/my_coords/spacing/dy"]= 1.0;
mesh["coordsets/my_coords/spacing/dz"]= 1.0;

mesh["topologies/my_topo/type"] = "uniform";
mesh["topologies/my_topo/coordset"] = "my_coords";
\end{lstlisting}
\caption{\label{ex:blueprint}An example of a specifying a $10^3$ uniform grid in Blueprint.}
\end{figure}
