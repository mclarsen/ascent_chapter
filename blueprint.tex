
%
In-memory sharing of scientific data is an important use case,
certainly for in situ visualization, but also for computational science in general.
%
One barrier to in-memory sharing is how scientific data is represented.
%
If a simulation code hands an array to an in situ visualization routine, then the routine
needs to know if this array is a scalar field, coordinate information, connectivity information
for an unstructured mesh, etc.
%
Conduit is not helpful in addressing this barrier,
as it provides a foundation for sharing data, but it provides no guidance on
how to arrange
data into a data model.
%
This is where Blueprint comes in.
%Unfortunately, Conduit
%There are many possible options, and Conduit
%Achieving this sharing can be difficult, as
%scientific data can be represented in myriad ways, creating a burden in mapping
%Conduit supports each of
%these layouts.
%
%The flexibility of Conduit Nodes allows it to be used to represent a
%wide range of scientific data.
%
%While this flexibility can be positive, it also can be negative ---
%unconstrained, this can lead to
%many application-specific choices for common types of data
%which in turn can be a barrier to
%sharing between applications.


The goal of Blueprint is to facilitate a set of shared higher-level
conventions for using Conduit Nodes to hold common simulation data structures.
%
Blueprint is able to describe a wide range of scientific data.
%
Its design emerged through two efforts.
%
First, by surveying existing
projects that provide scientific mesh-related APIs including: ADIOS, Damaris,
EAVL, MFEM, Silo, VTK, VTKm, and Xdmf.
%
In this way, lessons learned from previous taxonomies and concrete scientific data models
have been incorporated in Blueprint.
%
Second, through discussion with simulation application teams.
%
These discussions were about general in-memory sharing (i.e., including code-to-code sharing),
but have proven to be useful for in situ visualization use cases as well.
%
Finally,
Blueprint's mesh conventions are not a replacement for existing mesh data models or APIs.
%
Instead, Blueprint is trying to provide a comprehensive-but-small set of options for
describing meshes in-core that simplifies
the process of adapting data to several existing mesh-aware APIs.



%Many taxonomies and concrete mesh data models have been developed to allow
%computational meshes to be used in software.
%
%Blueprint’s conventions for representing mesh data were formed by negotiating
%with simulation application teams at LLNL and from
%

Blueprint uses four concepts to describe meshes:
\begin{itemize}
  \item \textbf{Coordinate Sets} describe coordinate systems in 1D, 2D, and 3D,
using Cartesian, spherical, or cylindrical frames of reference.
%
Blueprint supports not only explicit coordinate sets, but also implicit representations,
such as uniform and rectilinear.
  \item \textbf{Topologies} describe the topological structure of the mesh elements.
%
As with coordinate sets, both implicit (e.g., uniform) and explicit
(e.g., completely unstructured) topologies are supported.
  \item \textbf{Fields} describe the data associated with the mesh.
Fields can be associated with vertices or with elements, and
can be scalars or have multiple components.
%
Additionally, Blueprint supports the description of high-order
topologies and fields, which are becoming increasingly common.
  \item \textbf{Domain Decomposition Information} is important for distributed-memory applications.
%
In these cases, the mesh is typically distributed among MPI tasks, into ``domains,'' and
many algorithms need information at the boundaries of the domains to proceed.
%
There are many ways to describe how domain's abut, and Blueprint's domain decomposition information
allows for capturing the specifics.
\end{itemize}
A Blueprint data set minimally needs a topology and a coordinate system,
but Blueprint supports having any number of topologies and coordinate
systems, as well as any number of fields.
%


Listing~\ref{ex:blueprint} shows a Blueprint description of a uniform mesh.
%
Blueprint's form follows Conduit's key-value approach.
%
That said, the challenge in learning Blueprint is in learning the reserved words associated
with concepts.
%
In the figure's description of the coordinate system, there are many reserved words:
coordsets, type, dims,  i, j, k, and uniform.
%
Each of these have the expected meaning.
%
The figure also creates a variable, my\_coords.
%
Blueprint does provide help in setting a data set through its protocols ---
methods to verify if a Conduit Node instance conforms to known conventions.


\begin{lstlisting}[language=C++,caption={\label{ex:blueprint}An example of a specifying a $10^3$ uniform grid in Blueprint.}]
Node mesh;
// 10x10x10 uniform coordinate system
mesh["coordsets/my_coords/type"]="uniform";
mesh["coordsets/my_coords/dims/i"] = 10;
mesh["coordsets/my_coords/dims/j"] = 10;
mesh["coordsets/my_coords/dims/k"] = 10;

// optional origin
mesh["coordsets/my_coords/origin/x"]= -10;
mesh["coordsets/my_coords/origin/y"]= -10;
mesh["coordsets/my_coords/origin/z"]= -10;

mesh["coordsets/my_coords/spacing/dx"]= 1.0;
mesh["coordsets/my_coords/spacing/dy"]= 1.0;
mesh["coordsets/my_coords/spacing/dz"]= 1.0;

mesh["topologies/my_topo/type"] = "uniform";
mesh["topologies/my_topo/coordset"] = "my_coords";
\end{lstlisting}

