
The flexibility of the Conduit Node allows it to be used to represent a
wide range of scientific data.
%
Unconstrained, this flexibly can lead to
many application specific choices for common types of data that could
potentially be shared between applications.

The goal of Blueprint is to help facilitate a set of shared higher-level
conventions for using Conduit Nodes to hold common simulation data structures.
%
The Blueprint library in Conduit provides methods to verify if a Conduit
Node instance conforms to known conventions, which we call protocols.
%
It also provides property and transform methods that can be used on conforming Nodes.

Many taxonomies and concrete mesh data models have been developed to allow
computational meshes to be used in software.
%
Blueprint’s conventions for representing mesh data were formed by negotiating
with simulation application teams at LLNL and from a survey of existing
projects that provide scientific mesh-related APIs including: ADIOS, Damaris,
EAVL, MFEM, Silo, VTK, VTKm, and Xdmf.
%
Blueprint’s mesh conventions are not a replacement for existing mesh data
models or APIs.
%
Our explicit goal is to outline a comprehensive, but small set of options
for describing meshes in-core that simplifies the process of adapting data
to several existing mesh-aware APIs.

Blueprint covers a wide range of mesh desriptions, and
Blueprint uses four concepts to describe meshes:

\begin{itemize}
  \item Coordinate Sets
  \item Topologies
  \item Fields
  \item Domain Decomposition Information
\end{itemize}

Coordinate sets described coordinate systems in 1D, 2D, and 3D, and
can be specified in cartesian, spherical, or cyclidrical frames of reference.
%
In additon to explicit coordinate sets, compact implicit representations,
such as uniform and rectilinear, are also supported.
%
Topologies describe the topological structure of the mesh elements.
%
As with coordinate sets, both implicit(e.g., uniform) and explicit
(e.g., completely unstructured)topologies are supported.
%
Fields describe the data associtate with the mesh, and Blueprint supports
fields associated with verticies or elements.
%
Fields can be scalars or have multiple components.
%
Additionally, Blueprint supports the description of high-order
topologies and fields, which are becoming increasingly common.

A Blueprint data set minimally needs a topology and a coordinate system,
but Blueprint supports having any number of topologies and coordinate
systems.
%
Figure~\ref{ex:blueprint} shows a Blueprint description of a uniform mesh.

\begin{figure}
\begin{lstlisting}[language=C++]
Node mesh;
// 10x10x10 uniform cooridinate system
mesh["coordsets/my_coords/type"]="uniform";
mesh["coordsets/my_coords/dims/i"] = 10;
mesh["coordsets/my_coords/dims/j"] = 10;
mesh["coordsets/my_coords/dims/k"] = 10;

// optional origin
mesh["coordsets/my_coords/origin/x"]= -10;
mesh["coordsets/my_coords/origin/y"]= -10;
mesh["coordsets/my_coords/origin/z"]= -10;

mesh["coordsets/my_coords/spacing/dx"]= 1.0;
mesh["coordsets/my_coords/spacing/dy"]= 1.0;
mesh["coordsets/my_coords/spacing/dz"]= 1.0;

mesh["topologies/my_topo/type"] = "uniform";
mesh["topologies/my_topo/coordset"] = "my_coords";
\end{lstlisting}
\caption{\label{ex:blueprint}An example of a specifying a $10^3$ uniform grid in Blueprint.}
\end{figure}
